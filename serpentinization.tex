\documentclass[a4paper,14pt]{extarticle}
\usepackage{chemfig}
\usepackage{hyperref}
\usepackage{amsmath}
\usepackage{booktabs}
\usepackage[show]{ed}
\def\mhyphen{{\hbox{-}}}
\def\H{{\rm H}}
\def\O{{\rm O}}
\def\M{{\rm M}}
\def\C{{\rm C}}
\def\Ca{{\rm Ca}}
\def\Fe{{\rm Fe}}
\def\Mg{{\rm Mg}}
\def\Mn{{\rm Mn}}
\def\Ni{{\rm Ni}}
\def\Si{{\rm Si}}
\def\II{{\rm II}}
\def\III{{\rm III}}
\begin{document}

%%%%%%%%%%%%%%%%%%%%%%%%%%%%%%%%%%%%%%%%%%%%%%%%%%%%%%%%%%%%

\section{Introduction}

Olivine\cite{olivine} is a very abundant group of minerals (neso-silicates\cite{orthosilicate}) (present in Mafic rock,
e.g. Basalt) of the form
\[
    \M_2\Si\O_4,
\]
where $\M$ is one of the following metal divalent cations (group 2 of the periodic table).  The
elements and their associated mineral name are:
\begin{itemize}
    \itemsep=0em
    \item\Mg: Fosterite (also known as Peridot or Chrysolite). Melting point ca. $1900\ {}^\circ\C$.
    \item$\Fe^{\II}$: Fayalite. Melting point ca. $1200\ {}^\circ\C$.
    \item$\Mn^{\II}$: Tephroite
\end{itemize}
Mafic\cite{mafic} rock is a form of igneous rock (i.e.\ formed from lava) that contains combinations of Fosterite and Fayalite in
various proportions.  Only $\Fe^{\II}$ ions can form olivine, $\Fe^{\III}$ ions cannot.

As a comparison, $\H_4\Si\O_4$ (also known as $\Si{(\O\H)}_4$) is a form of Hydrated Silica\cite{hydrated_silica} formed
by the reaction
\[
    \Si\O_2 + 2\, \H_2\O \rightarrow \H_4\Si\O_4
\]
The mineral Quartz\cite{quartz} consists of a framework of Silicate atoms $\Si\O_4$ in a tetrahedal structure with each
oxygen being shared between two Silicate atoms. The overall chemical composition of Quartz is therefore $\Si\O_2$.


\section{Weathering}
Olivine is not stable on the surface and will (over geological time scales) spontaneously react with e.g.\ carbondioxide
in the following exothermic reaction:
\[
    \Mg_2\Si\O_4 + 2\,\C\O_2 \rightarrow 2\,\Mg\C\O_3 + \Si\O_2 + 90\, {\rm kJ/mole},
\]
where Fosterite decomposes to Magnesite and Silica.

\section{Serpentinization}

Serpentinization\cite{serpentinization} is a reaction between rock (e.g.\ ferromagnesian minerals such as Olivine) and
water that produces hydrogen gas (and serpentinite).
\[
    3\, \Fe^{\II}_2\Si \O_4 + 2\,\H_2\O \rightarrow 2\,\Fe^{\III}_3\O_4 + 3\,\Si \O_2 + 2\,\H_2
\]

It is based on the Schikorr reaction\cite{schikorr_reaction}
\[
    3\, \Fe{(\O\H)}_2 \rightarrow \Fe_3\O_4 + 2\,\H_2\O + \H_2
\]
which in turn is based on Iron (II,III) oxide\cite{iron_ii_iii_oxide}
\[
    \Fe^{\II}\O + \Fe_2^{\III}\O_3 \rightarrow \Fe_3\O_4
\]





\begin{thebibliography}{9}
    \bibitem{olivine}           \href{https://en.wikipedia.org/wiki/Olivine}{Olivine}
    \bibitem{orthosilicate}     \href{https://en.wikipedia.org/wiki/Orthosilicate}{Orthosilicate}
    \bibitem{mafic}             \href{https://en.wikipedia.org/wiki/Mafic}{Mafic}
    \bibitem{hydrated_silica}   \href{https://en.wikipedia.org/wiki/Hydrated_silica}{Hydrated Silica}
    \bibitem{quartz}            \href{https://en.wikipedia.org/wiki/Quartz}{Quartz}
    \bibitem{serpentinization}  \href{https://en.wikipedia.org/wiki/Serpentinization}{Serpentinization}
    \bibitem{schikorr_reaction} \href{https://en.wikipedia.org/wiki/Schikorr_reaction}{Schikorr reaction}
    \bibitem{iron_ii_iii_oxide} \href{https://en.wikipedia.org/wiki/Iron(II,III)_oxide}{Iron (II,III) oxide}
\end{thebibliography}

\end{document}

