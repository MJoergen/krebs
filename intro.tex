\documentclass{article}
\usepackage{chemfig}
\usepackage{hyperref}
\begin{document}

%%%%%%%%%%%%%%%%%%%%%%%%%%%%%%%%%%%%%%%%%%%%%%%%%%%%%%%%%%%%

\section{Introduction}

This document is about the bio-chemical processes happening in cells during metabolism and
respiration. The main entity in this process is the Krebs cycle.

The organisation is as follows: In section~(\ref{sec_prerequisites}) we begin by
establishing some facts from organic chemistry. Then in section~(\ref{sec_players})
we list the main characters taking place in this chemical dance.

%%%%%%%%%%%%%%%%%%%%%%%%%%%%%%%%%%%%%%%%%%%%%%%%%%%%%%%%%%%%

\section{Prerequisites}\label{sec_prerequisites}

Organic compounds are named after certain ``functional groups'' in the atomic structure.

\subsection{Alkane}

An alkane consists of carbon and hydrogen only, and with only single carbon-carbon bond.
They have the general chemical formula $C_{n}H_{2n+2}$. They are {\em saturated\/}
due to the lack of double bonds, and hence less reactive.

Examples of alkanes are methane ($CH_4$) and ethane ($C_2H_6$).  Methane is produced by
methanogenic bacteria.

\subsection{Alkene}

An alkene consists of carbon and hydrogen only, and with one carbon-carbon double bond.
They have the general chemical formula $C_{n}H_{2n}$.

Examples of alkanes are ethylene ($C_2H_4$) and propylene ($C_3H_6$).

\subsection{Aldehyde}

\chemfig{R-C (=[1]O) (-[7]H)}

The functional group (O=C---H) is called {\em aldehyde group\/} but also sometimes {\em
formyl group}.

Examples of aldehydes are: formaldehyde, acetaldehyde, glucose. However, in aqueous
solution only a tiny fraction of glucose exists as the aldehyde.

\subsection{carboxylic acid}
\chemfig{R-C (=[1]O) (-[7]OH)}

\subsection{ester}
\chemfig{R-C (=[1]O) (-[7]OR')}

\subsection{ketone}
\chemfig{R-C (=[1]O) (-[7]R')}

%%%%%%%%%%%%%%%%%%%%%%%%%%%%%%%%%%%%%%%%%%%%%%%%%%%%%%%%%%%%

\section{The players}\label{sec_players}

\subsection{Carboxylic acids}
\[
C_2H_4O_2
\]
\chemfig{H-C (-[2]H) (-[6]H)-C (=[1]O)-[7]{O^-}}

%%%%%%%%%%%%%%%%%%%%%%%%%%%%%%%%%%%%%%%%%%%%%%%%%%%%%%%%%%%%

\section{Reactions}\label{sec_reactions}

bla bla

%%%%%%%%%%%%%%%%%%%%%%%%%%%%%%%%%%%%%%%%%%%%%%%%%%%%%%%%%%%%

\section{Notes}\label{sec_notes}

Writing chemical formulae with chemfig is straightforward.

This is based on \href{https://www.overleaf.com/learn/latex/Chemistry_formulae}{chemfig}.

Wikipedia links:
\href{https://en.wikipedia.org/wiki/Biological_carbon_fixation}{Biological carbon fixation}.
\end{document}

