\documentclass{article}
\usepackage{chemfig}
\usepackage{hyperref}
\begin{document}

%%%%%%%%%%%%%%%%%%%%%%%%%%%%%%%%%%%%%%%%%%%%%%%%%%%%%%%%%%%%

\section{Introduction}

This document is about the bio-chemical processes happening in cells during metabolism and
respiration. The main entity in this process is the Krebs cycle.

The reverse Krebs cycle is a sequence of chemical reactions that are used by some bacteria
to produce carbon compounds from carbon dioxide and water by the use of energy-rich
reducing agents as electron donors.  Where the Krebs cycle takes carbohydrates and
oxidizes them to CO2 and water, the reverse cycle takes $CO_2$ and $H_2O$ to make carbon
compounds. This process is used by some bacteria (such as Aquificota) to synthesize carbon
compounds, sometimes using hydrogen, sulfide, or thiosulfate as
\href{https://en.wikipedia.org/wiki/Electron_donor}{electron donors}\footnote{An
electron donor is a chemical entity that donates electrons to another compound. It is a
\href{https://en.wikipedia.org/wiki/Reducing_agent}{reducing agent} that, by virtue of
its donating electrons, is itself oxidized in the
process. The electron donating power of a donor molecule is measured by its ionization
potential which is the energy required to remove an electron from the highest occupied
molecular orbital (HOMO). The driving forces for electron donor and acceptor behavior in
chemistry is based on the concepts of electropositivity (for donors) and electronegativity
(for acceptors) of atomic or molecular entities.}.

There are three enzymes specific to the reductive Krebs cycle: citrate lyase,
fumarate reductase, and α-ketoglutarate synthase.

The splitting of citric acid to oxaloacetate and acetate is catalyzed by citrate lyase,
rather than the reverse reaction of citrate synthase. Succinate dehydrogenase is
replaced by fumarate reductase and α-ketoglutarate synthase replaces α-ketoglutarate
dehydrogenase.

The conversion of succinate to 2-oxoglutarate is also different. In the oxidative reaction
this step is coupled to the reduction of NADH\@. However, the oxidation of 2-oxoglutarate to
succinate is so energetically favorable, that NADH lacks the reductive power to drive the
reverse reaction. In the reverse Krebs cycle, this reaction has to use a reduced low potential
\href{https://en.wikipedia.org/wiki/Ferredoxin}{ferredoxin}.

The organisation is as follows: In section~(\ref{sec_prerequisites}) we begin by
establishing some facts from organic chemistry. Then in section~(\ref{sec_players})
we list the main characters taking place in this chemical dance.

\subsection{Redox reaction}
In their pre-reaction states, reducers (i.e. {\em electron donors\/}) have extra electrons
(that is, they are by themselves reduced) and oxidizers (i.e. {\em electron acceptors\/})
lack electrons (that is, they are by themselves oxidized). This is commonly expressed in
terms of their oxidation states. An agent's oxidation state describes its degree of loss
of electrons, where the higher the oxidation state then the fewer electrons it has. So
initially, prior to the reaction, a reducing agent is typically in one of its lower
possible oxidation states; its oxidation state increases during the reaction while that of
the oxidizer decreases. Thus in a redox reaction, the agent whose oxidation state
increases, that ``loses/donates electrons'', that ``is oxidized'', and that ``reduces'' is
called the reducer or reducing agent, while the agent whose oxidation state decreases,
that ``gains/accepts/receives electrons'', that ``is reduced'', and that ``oxidizes'' is
called the oxidizer or oxidizing agent.

For example, consider the overall reaction for aerobic cellular respiration:
\[
    C_6H_{12}O_6(s) + 6O_2(g) \rightarrow 6CO_2(g) + 6H_2O(l)
\]
The oxygen ($O_2$) is being reduced, so it is the oxidizing agent. The glucose
($C_6H_{12}O_6$) is being oxidized, so it is the reducing agent.

\subsection{Electron donor = reducing agent}

Strong reducing agents easily lose (or donate) electrons. An atom with a relatively large
atomic radius tends to be a better reductant. In such species, the distance from the
nucleus to the valence electrons is so long that these electrons are not strongly
attracted. These elements tend to be strong reducing agents. Good reducing agents tend to
consist of atoms with a low electronegativity, which is the ability of an atom or molecule
to attract bonding electrons, and species with relatively small ionization energies serve
as good reducing agents too.

The measure of a material's ability to reduce is known as its
\href{https://en.wikipedia.org/wiki/Reduction_potential} reduction potential.


%%%%%%%%%%%%%%%%%%%%%%%%%%%%%%%%%%%%%%%%%%%%%%%%%%%%%%%%%%%%

\pagebreak
\section{Prerequisites}\label{sec_prerequisites}

Organic compounds are named after certain ``functional groups'' in the atomic structure.

\subsection{Alkane}

An alkane consists of carbon and hydrogen only, and with only single carbon-carbon bond.
They have the general chemical formula $C_{n}H_{2n+2}$. They are {\em saturated\/}
due to the lack of double bonds (and hence less reactive?).

Examples of alkanes are methane ($CH_4$), ethane ($C_2H_6$), and octane ($C_{8}H_{18}$).
Methane is produced by methanogenic bacteria.

\subsection{Alkene}

An alkene consists of carbon and hydrogen only, and with one carbon-carbon double bond.
They have the general chemical formula $C_{n}H_{2n}$. The double bond is stronger than
a single bond, but not twice as strong.

Alkenes react in many addition reactions, which occur by opening up the double-bond.

Examples of alkanes are ethylene ($C_2H_4$) and propylene ($C_3H_6$).

\subsection{Aldehyde}

\chemfig{R-C (=[1]O) (-[7]H)}

The functional group (O=C---H) is called {\em aldehyde group\/} but also sometimes {\em
formyl group}.

Examples of aldehydes are: formaldehyde, acetaldehyde, glucose. However, in aqueous
solution only a tiny fraction of glucose exists as the aldehyde.

\subsection{carboxylic acid}
\chemfig{R-C (=[1]O) (-[7]OH)}

Carboxylic acids are polar. Because they are both hydrogen-bond acceptors (the {\em
carbonyl\/} $–C=O$) and hydrogen-bond donors (the {\em hydroxyl\/} $–OH$), they also
participate in hydrogen bonding. Together, the hydroxyl and carbonyl group form the
functional group {\em carboxyl}.

Most carboxylic acids can be reduced to alcohols by hydrogenation.

Examples are formic acid (formate) and acetic acid (acetate).

\subsection{ester}
\chemfig{R-C (=[1]O) (-[7]OR')}

\subsection{ketone}
\chemfig{R-C (=[1]O) (-[7]R')}

\subsection{carbohydrate}
A carbohydrate is a biomolecule consisting of carbon (C), hydrogen (H) and oxygen (O)
atoms, usually with a hydrogen–oxygen atom ratio of 2:1 (as in water) and thus with the
empirical formula $C_m{(H_2O)}_n$ (where m may or may not be different from n), which does
not mean the H has covalent bonds with O (for example with $CH_2O$, H has a covalent bond
with C but not with O). However, not all carbohydrates conform to this precise
stoichiometric definition (e.g., uronic acids, deoxy-sugars such as fucose), nor are all
chemicals that do conform to this definition automatically classified as carbohydrates
(e.g.\ formaldehyde and acetic acid).

%%%%%%%%%%%%%%%%%%%%%%%%%%%%%%%%%%%%%%%%%%%%%%%%%%%%%%%%%%%%

\pagebreak
\section{The players}\label{sec_players}

\subsection{Aldehydes}

\subsubsection{Formaldehyde}
\chemfig{H-C (=[1]O) (-[7]H)}

\begin{itemize}
    \item Chemical formula $CH_2O$.
    \item Standard enthalpy of formation: -108.7 kJ/mol.
\end{itemize}

\subsubsection{Acetaldehyde}
\chemfig{H-C (-[2]H) (-[6]H) -C (=[1]O) (-[7]H)}

\begin{itemize}
    \item Chemical formula $C_2H_4O$.
    \item Standard enthalpy of formation: -192.2 kJ/mol.
\end{itemize}

\subsection{Carboxylic acids}

In the following, the carboxylate anions are given, a result of deprotonation.

\subsubsection{Formate}
\chemfig{H-C (=[1]O)-[7]{O^-}}
\begin{itemize}
    \item Chemical formula $CHO_2^-$.
    \item Standard enthalpy of formation (of formic acid): -425.0 kJ/mol.
\end{itemize}
The two oxygen atoms are equivalent and bear a partial negative charge. The remaining C-H
bond is not acidic.

\subsubsection{Acetate}
\chemfig{H-C (-[2]H) (-[6]H)-C (=[1]O)-[7]{O^-}}
\begin{itemize}
    \item Chemical formula $C_2H_3O_2^-$.
    \item Standard enthalpy of formation:
\end{itemize}

\subsubsection{Pyruvate}
\chemfig{H-C (-[2]H) (-[6]H)-C (=[2]O)-C (=[1]O) -[7]{O^-}}
\begin{itemize}
    \item Chemical formula $C_3H_3O_3^-$.
    \item Standard enthalpy of formation:
\end{itemize}

\subsubsection{Lactate}
\chemfig{H-C (-[2]H) (-[6]H)-C (-[2]OH) (-[6]H) -C (=[1]O) -[7]{O^-}}
\begin{itemize}
    \item Chemical formula $C_3H_5O_3^-$.
    \item Standard enthalpy of formation:
\end{itemize}

\subsubsection{Glycerate}
\chemfig{HO-C (-[2]H) (-[6]H)-C (-[2]OH) (-[6]H) -C (=[1]O) -[7]{O^-}}
\begin{itemize}
    \item Chemical formula $C_3H_5O_4^-$.
    \item Standard enthalpy of formation:
\end{itemize}

\subsubsection{Acetoacetate}
\chemfig{C (-[2]H) (-[4]H) (-[6]H) -C (=[6]O) -C (-[2]H) (-[6]H) -C (=[1]O) -[7]{O^-}}
\begin{itemize}
    \item Chemical formula $C_4H_5O_3^-$.
    \item Standard enthalpy of formation:
\end{itemize}

\subsubsection{Fumarate}
\chemfig{C (=[3]O) (-[5]HO) -C (-[6]H) =C (-[6]H) -C (=[1]O) -[7]{O^-}}
\begin{itemize}
    \item Chemical formula $C_4H_3O_4^-$.
    \item Standard enthalpy of formation:
\end{itemize}

\subsubsection{Succinate}
\chemfig{C (=[3]O) (-[5]HO) -C (-[2]H) (-[6]H) -C (-[2]H) (-[6]H) -C (=[1]O) -[7]{O^-}}
\begin{itemize}
    \item Chemical formula $C_4H_5O_4^-$.
    \item Standard enthalpy of formation:
\end{itemize}

\subsubsection{Malate}
\chemfig{C (=[3]O) (-[5]HO) -C (-[2]H) (-[6]H) -C (-[2]H) (-[6]OH) -C (=[1]O) -[7]{O^-}}
\begin{itemize}
    \item Chemical formula $C_4H_5O_5^-$.
    \item Standard enthalpy of formation:
\end{itemize}

\subsubsection{Oxaloacetate}
\chemfig{C (=[3]O) (-[5]HO) -C (=[6]O) -C (-[2]H) (-[6]H) -C (=[1]O) -[7]{O^-}}
\begin{itemize}
    \item Chemical formula $C_4H_3O_5^-$.
    \item Standard enthalpy of formation:
\end{itemize}

\subsubsection{Glutarate}
\chemfig{C (=[3]O) (-[5]{O^-}) -C (-[2]H) (-[6]H) -C (-[2]H) (-[6]H) -C (-[2]H) (-[6]H) -C (=[1]O) -[7]{O^-}}
\begin{itemize}
    \item Chemical formula $C_5H_6O_4^{2-}$.
    \item Standard enthalpy of formation:
\end{itemize}

\subsubsection{Alpha-ketoglutarate}
\chemfig{C (=[3]O) (-[5]{O^-}) -C (=[6]O) -C (-[2]H) (-[6]H) -C (-[2]H) (-[6]H) -C (=[1]O) -[7]{O^-}}
\begin{itemize}
    \item Chemical formula $C_5H_4O_5^{2-}$.
    \item Standard enthalpy of formation:
\end{itemize}

\subsubsection{Oxalosuccinate}
\chemfig{C (=[3]O) (-[5]{O^-}) -C (-[2]H) (-[6]H)  -C (-[2]H) (-[6] C (=[5]O) (-[7]{O^-})) -C (=[2]O) -C (=[1]O) -[7]{O^-}}
\begin{itemize}
    \item Chemical formula $C_6H_3O_7^{3-}$.
    \item Standard enthalpy of formation:
\end{itemize}

\subsubsection{Citrate}
\chemfig{C (=[3]O) (-[5]{O^-}) -C (-[2]H) (-[6]H) -C (-[2]OH) (-[6] C (=[5]O) (-[7]{O^-}))
-C (-[2]H) (-[6]H) -C (=[1]O) -[7]{O^-}}
\begin{itemize}
    \item Chemical formula $C_6H_5O_7^{3-}$.
    \item Standard enthalpy of formation:
\end{itemize}


\subsubsection{Ascorbate}
\begin{itemize}
    \item Chemical formula $C_6H_8O_6$.
\end{itemize}
The ascorbate ion is the predominant species at typical biological pH values. It is a mild
reducing agent and antioxidant. It is oxidized with loss of one electron to form a radical
cation and then with loss of a second electron to form dehydroascorbic acid. It typically
reacts with oxidants of the reactive oxygen species, such as the hydroxyl radical.

Ascorbic acid is special because it can transfer a single electron, owing to the
resonance-stabilized nature of its own radical ion, called semidehydroascorbate. The net
reaction is:

\[
    RO + C_6H_7O_6^- \rightarrow RO^− + C_6H_7O_6 \rightarrow ROH + C_6H_6O_6
\]

On exposure to oxygen, ascorbic acid will undergo further oxidative decomposition to
various products including diketogulonic acid, xylonic acid, threonic acid and oxalic
acid.

Reactive oxygen species are damaging to animals and plants at the molecular level due to
their possible interaction with nucleic acids, proteins, and lipids. Sometimes these
radicals initiate chain reactions. Ascorbate can terminate these chain radical reactions
by electron transfer. The oxidized forms of ascorbate are relatively unreactive and do not
cause cellular damage.

However, being a good electron donor, excess ascorbate in the presence of free metal ions
can not only promote but also initiate free radical reactions, thus making it a
potentially dangerous pro-oxidative compound in certain metabolic contexts.

\subsection{Sugars}

\subsubsection{Glyceraldehyde}
\chemfig{C (=[3]O) (-[5]{H}) -C (-[2]OH) (-[6]H) -C (-[2]H) (-[6]OH) -H}
\begin{itemize}
    \item Chemical formula $C_3H_6O_3$.
    \item Standard enthalpy of formation:
\end{itemize}

\subsubsection{Dihydroxyacetone}
\chemfig{C (-[4]H) (-[2]H) (-[6]{OH}) -C (=[2]O) -C (-[2]H) (-[6]OH) -H}
\begin{itemize}
    \item Chemical formula $C_3H_6O_3$.
    \item Standard enthalpy of formation:
\end{itemize}

\subsubsection{Glucose}
\chemfig{C (=[3]O) (-[5]H) -C (-[2]OH) (-[6]H) -C (-[2]H) (-[6]OH) -C (-[2]OH) (-[6]H) -C
(-[2]OH) (-[6]H) -C (-[2]H) (-[6]OH) -H}
\begin{itemize}
    \item Chemical formula $C_6H_{12}O_6$.
    \item Standard enthalpy of formation:
\end{itemize}

\subsubsection{Fructose}
This consists of a combination of dihydroxyacetone and glyceraldehyde.

\chemfig{C (-[4]H) (-[2]H) (-[6]OH) -C (=[2]O) -C (-[2]H) (-[6]OH) -C (-[2]OH) (-[6]H) -C
(-[2]OH) (-[6]H) -C (-[2]H) (-[6]OH) -H}
\begin{itemize}
    \item Chemical formula $C_6H_{12}O_6$.
    \item Standard enthalpy of formation:
\end{itemize}


%%%%%%%%%%%%%%%%%%%%%%%%%%%%%%%%%%%%%%%%%%%%%%%%%%%%%%%%%%%%

\pagebreak
\section{Reactions}\label{sec_reactions}

Energy is released from $ATP$ by the following reaction.
\[
    {ATP}^{4-} + H_2O \rightarrow {ADP}^{3-} + {HPO_4}^{2-} + H^+
\]

Glycolysis:
\[
    C_6H_{12}O_6 + 2{NAD}^+ + 2{ADP}^{3-} + 2{HPO_4}^{2-} \rightarrow
    2{C_3H_3O_3}^- + 2NADH + 2H^+ + 2{ATP}^{4-} + 2H_2O
\]

\subsection{The Krebs cycle}
The following reaction converts pyruvate into oxaloacetate, and is catalyzed by Pyruvate
Carboxylase.
\[
    {ATP}^{4-} + {C_3H_3O_3}^- + {HCO_3}^- \rightarrow {ADP}^{3-} + {HPO_4}^{2-} + {C_4H_3O_5}^-
\]

\subsection{The Wood–Ljungdahl pathway}
\href{https://en.wikipedia.org/wiki/Wood-Ljungdahl_pathway}{The Wood-Ljungdahl pathway} is
a set of biochemical reactions used by some bacteria. It is
also known as the reductive acetyl-coenzyme A (Acetyl-CoA) pathway. This pathway
enables these organisms to use hydrogen as an electron donor, and carbon dioxide as an
electron acceptor and as a building block for biosynthesis.


%%%%%%%%%%%%%%%%%%%%%%%%%%%%%%%%%%%%%%%%%%%%%%%%%%%%%%%%%%%%

\pagebreak
\section{Notes}\label{sec_notes}

We need something about ``Standard enthalpy of formation''.

Writing chemical formulae with chemfig is straightforward.

This is based on \href{https://www.overleaf.com/learn/latex/Chemistry_formulae}{chemfig}.

Wikipedia links:
\href{https://en.wikipedia.org/wiki/Biological_carbon_fixation}{Biological carbon fixation}.
\end{document}

