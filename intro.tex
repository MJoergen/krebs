\documentclass{article}
\usepackage{chemfig}
\usepackage{hyperref}
\usepackage{amsmath}
\def\mhyphen{{\hbox{-}}}
\begin{document}

%%%%%%%%%%%%%%%%%%%%%%%%%%%%%%%%%%%%%%%%%%%%%%%%%%%%%%%%%%%%

\section{Introduction}

This document is about the bio-chemical processes happening in cells during metabolism and
respiration. The main entity in this process is the Krebs cycle.

The reverse Krebs cycle is a sequence of chemical reactions that are used by some bacteria
to produce carbon compounds from carbon dioxide and water by the use of energy-rich
reducing agents as electron donors.  Where the Krebs cycle takes carbohydrates and
oxidizes them to CO2 and water, the reverse cycle takes $CO_2$ and $H_2O$ to make carbon
compounds. This process is used by some bacteria (such as Aquificota) to synthesize carbon
compounds, sometimes using hydrogen, sulfide, or thiosulfate as
\href{https://en.wikipedia.org/wiki/Electron_donor}{electron donors}\footnote{An
electron donor is a chemical entity that donates electrons to another compound. It is a
\href{https://en.wikipedia.org/wiki/Reducing_agent}{reducing agent} that, by virtue of
its donating electrons, is itself oxidized in the
process. The electron donating power of a donor molecule is measured by its ionization
potential which is the energy required to remove an electron from the highest occupied
molecular orbital (HOMO). The driving forces for electron donor and acceptor behavior in
chemistry is based on the concepts of electropositivity (for donors) and electronegativity
(for acceptors) of atomic or molecular entities.}.

There are three enzymes specific to the reductive Krebs cycle: citrate lyase,
fumarate reductase, and α-ketoglutarate synthase.

The splitting of citric acid to oxaloacetate and acetate is catalyzed by citrate lyase,
rather than the reverse reaction of citrate synthase. Succinate dehydrogenase is
replaced by fumarate reductase and α-ketoglutarate synthase replaces α-ketoglutarate
dehydrogenase.

The conversion of succinate to 2-oxoglutarate is also different. In the oxidative reaction
this step is coupled to the reduction of NADH\@. However, the oxidation of 2-oxoglutarate to
succinate is so energetically favorable, that NADH lacks the reductive power to drive the
reverse reaction. In the reverse Krebs cycle, this reaction has to use a reduced low potential
\href{https://en.wikipedia.org/wiki/Ferredoxin}{ferredoxin}.

The organisation is as follows: In section~(\ref{sec_prerequisites}) we begin by
establishing some facts from organic chemistry. Then in section~(\ref{sec_players})
we list the main characters taking place in this chemical dance.

\subsection{Redox reaction}
In their pre-reaction states, reducers (i.e. {\em electron donors\/}) have extra electrons
(that is, they are by themselves reduced) and oxidizers (i.e. {\em electron acceptors\/})
lack electrons (that is, they are by themselves oxidized). This is commonly expressed in
terms of their oxidation states. An agent's oxidation state describes its degree of loss
of electrons, where the higher the oxidation state then the fewer electrons it has. So
initially, prior to the reaction, a reducing agent is typically in one of its lower
possible oxidation states; its oxidation state increases during the reaction while that of
the oxidizer decreases. Thus in a redox reaction, the agent whose oxidation state
increases, that ``loses/donates electrons'', that ``is oxidized'', and that ``reduces'' is
called the reducer or reducing agent, while the agent whose oxidation state decreases,
that ``gains/accepts/receives electrons'', that ``is reduced'', and that ``oxidizes'' is
called the oxidizer or oxidizing agent.

For example, consider the overall reaction for aerobic cellular respiration:
\[
    C_6H_{12}O_6(s) + 6O_2(g) \rightarrow 6CO_2(g) + 6H_2O(l)
\]
The oxygen ($O_2$) is being reduced, so it is the oxidizing agent. The glucose
($C_6H_{12}O_6$) is being oxidized, so it is the reducing agent.

\subsection{Electron donor = reducing agent}

Strong reducing agents easily lose (or donate) electrons. An atom with a relatively large
atomic radius tends to be a better reductant. In such species, the distance from the
nucleus to the valence electrons is so long that these electrons are not strongly
attracted. These elements tend to be strong reducing agents. Good reducing agents tend to
consist of atoms with a low electronegativity, which is the ability of an atom or molecule
to attract bonding electrons, and species with relatively small ionization energies serve
as good reducing agents too.

The measure of a material's ability to reduce is known as its
\href{https://en.wikipedia.org/wiki/Reduction_potential} reduction potential.


%%%%%%%%%%%%%%%%%%%%%%%%%%%%%%%%%%%%%%%%%%%%%%%%%%%%%%%%%%%%

\pagebreak
\section{Prerequisites}\label{sec_prerequisites}

Organic compounds are named after certain ``functional groups'' in the atomic structure.

\subsection{Alkane}

An alkane consists of carbon and hydrogen only, and with only single carbon-carbon bond.
They have the general chemical formula $C_{n}H_{2n+2}$. They are {\em saturated\/}
due to the lack of double bonds (and hence less reactive?).

Examples of alkanes are methane ($CH_4$), ethane ($C_2H_6$), and octane ($C_{8}H_{18}$).
Methane is produced by methanogenic bacteria.

\subsection{Alkene}

An alkene consists of carbon and hydrogen only, and with one carbon-carbon double bond.
They have the general chemical formula $C_{n}H_{2n}$. The double bond is stronger than
a single bond, but not twice as strong.

Alkenes react in many addition reactions, which occur by opening up the double-bond.

Examples of alkanes are ethylene ($C_2H_4$) and propylene ($C_3H_6$).

\subsection{Aldehyde}

\chemfig{R-C (=[1]O) (-[7]H)}

The functional group (O=C---H) is called {\em aldehyde group\/} but also sometimes {\em
formyl group}.

Examples of aldehydes are: formaldehyde, acetaldehyde, glucose. However, in aqueous
solution only a tiny fraction of glucose exists as the aldehyde.

\subsection{carboxylic acid}
\chemfig{R-C (=[1]O) (-[7]OH)}

Carboxylic acids are polar. Because they are both hydrogen-bond acceptors (the {\em
carbonyl\/} $–C=O$) and hydrogen-bond donors (the {\em hydroxyl\/} $–OH$), they also
participate in hydrogen bonding. Together, the hydroxyl and carbonyl group form the
functional group {\em carboxyl}.

Most carboxylic acids can be reduced to alcohols by hydrogenation.

Examples are formic acid (formate) and acetic acid (acetate).

\subsection{ester}
\chemfig{R-C (=[1]O) (-[7]OR')}

They are formed when a carboxylic acid reacts via a condensation reaction (i.e.\ releasing
water) with an alcohol:
\[
    RCOOH + R'OH \rightarrow RCOOR' + H_2O
\]

Similarly, a phosphate ion $HPO_4^{2-}$ can react with an alcohol --- again via a
condensation reaction --- yielding a phosphoester link. In fact, a phophate ion can make
two phosphoester links, binding to two different alcohols.

\subsection{ketone}
\chemfig{R-C (=[1]O) (-[7]R')}

\subsection{carbohydrate}
A carbohydrate is a biomolecule consisting of carbon (C), hydrogen (H) and oxygen (O)
atoms, usually with a hydrogen–oxygen atom ratio of 2:1 (as in water) and thus with the
empirical formula $C_m{(H_2O)}_n$ (where m may or may not be different from n), which does
not mean the H has covalent bonds with O (for example with $CH_2O$, H has a covalent bond
with C but not with O). However, not all carbohydrates conform to this precise
stoichiometric definition (e.g., uronic acids, deoxy-sugars such as fucose), nor are all
chemicals that do conform to this definition automatically classified as carbohydrates
(e.g.\ formaldehyde and acetic acid).

%%%%%%%%%%%%%%%%%%%%%%%%%%%%%%%%%%%%%%%%%%%%%%%%%%%%%%%%%%%%

\pagebreak
\section{The players}\label{sec_players}

\subsection{Aldehydes}

\subsubsection{Formaldehyde}
\chemfig{H-C (=[1]O) (-[7]H)}

\begin{itemize}
    \item Chemical formula $CH_2O$.
    \item Standard enthalpy of formation: -108.7 kJ/mol.
\end{itemize}

\subsubsection{Acetaldehyde}
\chemfig{H-C (-[2]H) (-[6]H) -C (=[1]O) (-[7]H)}

\begin{itemize}
    \item Chemical formula $C_2H_4O$.
    \item Standard enthalpy of formation: -192.2 kJ/mol.
\end{itemize}

\subsection{Carboxylic acids}

In the following, the carboxylate anions are given, a result of deprotonation.

\subsubsection{Formate}
\chemfig{H-C (=[1]O)-[7]{O^-}}
\begin{itemize}
    \item Chemical formula $CHO_2^-$.
    \item Standard enthalpy of formation (of formic acid): -425.0 kJ/mol.
\end{itemize}
The two oxygen atoms are equivalent and bear a partial negative charge. The remaining C-H
bond is not acidic.

\subsubsection{Acetate}
\chemfig{H-C (-[2]H) (-[6]H)-C (=[1]O)-[7]{O^-}}
\begin{itemize}
    \item Chemical formula $C_2H_3O_2^-$.
    \item Standard enthalpy of formation:
\end{itemize}

\subsubsection{Pyruvate}
\chemfig{H-C (-[2]H) (-[6]H)-C (=[2]O)-C (=[1]O) -[7]{O^-}}
\begin{itemize}
    \item Chemical formula $C_3H_3O_3^-$.
    \item Standard enthalpy of formation:
\end{itemize}

\subsubsection{Lactate}
\chemfig{H-C (-[2]H) (-[6]H)-C (-[2]OH) (-[6]H) -C (=[1]O) -[7]{O^-}}
\begin{itemize}
    \item Chemical formula $C_3H_5O_3^-$.
    \item Standard enthalpy of formation:
\end{itemize}

\subsubsection{Glycerate}
\chemfig{HO-C (-[2]H) (-[6]H)-C (-[2]OH) (-[6]H) -C (=[1]O) -[7]{O^-}}
\begin{itemize}
    \item Chemical formula $C_3H_5O_4^-$.
    \item Standard enthalpy of formation:
\end{itemize}

\subsubsection{Acetoacetate}
\chemfig{C (-[2]H) (-[4]H) (-[6]H) -C (=[6]O) -C (-[2]H) (-[6]H) -C (=[1]O) -[7]{O^-}}
\begin{itemize}
    \item Chemical formula $C_4H_5O_3^-$.
    \item Standard enthalpy of formation:
\end{itemize}

\subsubsection{Fumarate}
\chemfig{C (=[3]O) (-[5]{O^-}) -C (-[6]H) =C (-[6]H) -C (=[1]O) -[7]{O^-}}
\begin{itemize}
    \item Chemical formula $C_4H_2O_4^{2-}$.
    \item Standard enthalpy of formation:
\end{itemize}

\subsubsection{Succinate}
\chemfig{C (=[3]O) (-[5]HO) -C (-[2]H) (-[6]H) -C (-[2]H) (-[6]H) -C (=[1]O) -[7]{O^-}}
\begin{itemize}
    \item Chemical formula $C_4H_5O_4^-$.
    \item Standard enthalpy of formation:
\end{itemize}

\subsubsection{Malate}
\chemfig{C (=[3]O) (-[5]HO) -C (-[2]H) (-[6]H) -C (-[2]H) (-[6]OH) -C (=[1]O) -[7]{O^-}}
\begin{itemize}
    \item Chemical formula $C_4H_5O_5^-$.
    \item Standard enthalpy of formation:
\end{itemize}

\subsubsection{Oxaloacetate}
\chemfig{C (=[3]O) (-[5]HO) -C (=[6]O) -C (-[2]H) (-[6]H) -C (=[1]O) -[7]{O^-}}
\begin{itemize}
    \item Chemical formula $C_4H_3O_5^-$.
    \item Standard enthalpy of formation:
\end{itemize}

\subsubsection{Glutarate}
\chemfig{C (=[3]O) (-[5]{O^-}) -C (-[2]H) (-[6]H) -C (-[2]H) (-[6]H) -C (-[2]H) (-[6]H) -C (=[1]O) -[7]{O^-}}
\begin{itemize}
    \item Chemical formula $C_5H_6O_4^{2-}$.
    \item Standard enthalpy of formation:
\end{itemize}

\subsubsection{Alpha-ketoglutarate}
\chemfig{C (=[3]O) (-[5]{O^-}) -C (=[6]O) -C (-[2]H) (-[6]H) -C (-[2]H) (-[6]H) -C (=[1]O) -[7]{O^-}}
\begin{itemize}
    \item Chemical formula $C_5H_4O_5^{2-}$.
    \item Standard enthalpy of formation:
\end{itemize}

\subsubsection{Oxalosuccinate}
\chemfig{C (=[3]O) (-[5]{O^-}) -C (-[2]H) (-[6]H)  -C (-[2]H) (-[6] C (=[5]O) (-[7]{O^-})) -C (=[2]O) -C (=[1]O) -[7]{O^-}}
\begin{itemize}
    \item Chemical formula $C_6H_3O_7^{3-}$.
    \item Standard enthalpy of formation:
\end{itemize}

\subsubsection{Citrate}
\chemfig{C (=[3]O) (-[5]{O^-}) -C (-[2]H) (-[6]H) -C (-[2]OH) (-[6] C (=[5]O) (-[7]{O^-}))
-C (-[2]H) (-[6]H) -C (=[1]O) -[7]{O^-}}
\begin{itemize}
    \item Chemical formula $C_6H_5O_7^{3-}$.
    \item Standard enthalpy of formation:
\end{itemize}
This is a tricarboxylic acid, because it has three carboxyl groups.

\subsubsection{Aconitate}
\chemfig{C (=[3]O) (-[5]{O^-}) -C (-[2]H) (-[6]H) -C (-[6] C (=[5]O) (-[7]{O^-}))
=C (-[6]H) -C (=[1]O) -[7]{O^-}}
\begin{itemize}
    \item Chemical formula $C_6H_3O_6^{3-}$.
    \item Standard enthalpy of formation:
\end{itemize}
This is a tricarboxylic acid, because it has three carboxyl groups.

\subsubsection{Isocitrate}
\chemfig{C (=[3]O) (-[5]{O^-}) -C (-[2]H) (-[6]H) -C (-[2]H) (-[6] C (=[5]O) (-[7]{O^-}))
-C (-[2]OH) (-[6]H) -C (=[1]O) -[7]{O^-}}
\begin{itemize}
    \item Chemical formula $C_6H_5O_7^{3-}$.
    \item Standard enthalpy of formation:
\end{itemize}
This is a tricarboxylic acid, because it has three carboxyl groups.

\subsubsection{Ascorbate}
\begin{itemize}
    \item Chemical formula $C_6H_8O_6$.
\end{itemize}
The ascorbate ion is the predominant species at typical biological pH values. It is a mild
reducing agent and antioxidant. It is oxidized with loss of one electron to form a radical
cation and then with loss of a second electron to form dehydroascorbic acid. It typically
reacts with oxidants of the reactive oxygen species, such as the hydroxyl radical.

Ascorbic acid is special because it can transfer a single electron, owing to the
resonance-stabilized nature of its own radical ion, called semidehydroascorbate. The net
reaction is:

\[
    RO + C_6H_7O_6^- \rightarrow RO^− + C_6H_7O_6 \rightarrow ROH + C_6H_6O_6
\]

On exposure to oxygen, ascorbic acid will undergo further oxidative decomposition to
various products including diketogulonic acid, xylonic acid, threonic acid and oxalic
acid.

Reactive oxygen species are damaging to animals and plants at the molecular level due to
their possible interaction with nucleic acids, proteins, and lipids. Sometimes these
radicals initiate chain reactions. Ascorbate can terminate these chain radical reactions
by electron transfer. The oxidized forms of ascorbate are relatively unreactive and do not
cause cellular damage.

However, being a good electron donor, excess ascorbate in the presence of free metal ions
can not only promote but also initiate free radical reactions, thus making it a
potentially dangerous pro-oxidative compound in certain metabolic contexts.

\subsection{Sugars}

\subsubsection{Glyceraldehyde}
\chemfig{C (=[3]O) (-[5]{H}) -C (-[2]OH) (-[6]H) -C (-[2]H) (-[6]OH) -H}
\begin{itemize}
    \item Chemical formula $C_3H_6O_3$.
    \item Standard enthalpy of formation:
\end{itemize}

\subsubsection{Dihydroxyacetone}
\chemfig{C (-[4]H) (-[2]H) (-[6]{OH}) -C (=[2]O) -C (-[2]H) (-[6]OH) -H}
\begin{itemize}
    \item Chemical formula $C_3H_6O_3$.
    \item Standard enthalpy of formation:
\end{itemize}

\subsubsection{Glucose}
\chemfig{C (=[3]O) (-[5]H) -C (-[2]OH) (-[6]H) -C (-[2]H) (-[6]OH) -C (-[2]OH) (-[6]H) -C
(-[2]OH) (-[6]H) -C (-[2]H) (-[6]OH) -H}
\begin{itemize}
    \item Chemical formula $C_6H_{12}O_6$.
    \item Standard enthalpy of formation:
\end{itemize}

\subsubsection{Fructose}
This consists of a combination of dihydroxyacetone and glyceraldehyde.

\chemfig{C (-[4]H) (-[2]H) (-[6]OH) -C (=[2]O) -C (-[2]H) (-[6]OH) -C (-[2]OH) (-[6]H) -C
(-[2]OH) (-[6]H) -C (-[2]H) (-[6]OH) -H}
\begin{itemize}
    \item Chemical formula $C_6H_{12}O_6$.
    \item Standard enthalpy of formation:
\end{itemize}


%%%%%%%%%%%%%%%%%%%%%%%%%%%%%%%%%%%%%%%%%%%%%%%%%%%%%%%%%%%%

\pagebreak
\section{Reactions}\label{sec_reactions}

Energy is released from $ATP$ by the following reaction splitting it into $ADP$ and a
phosphate $PO_3^-$ group.
\[
    {ATP}^{4-} + H_2O \leftrightarrow {ADP}^{3-} + {HPO_4}^{2-} + H^+
\]
Conversely, the opposite reaction is used to store energy, by adding a phosphate group to
$ADP$ turning it into $ATP$.

\subsection{Pyruvate decarboxylation}
Pyruvate produced by glycolysis still contains a lot of reducing power (check each of its
carbon atoms' oxidation state and compare it with carbon's oxidation state in $CO_2$). This
reducing power will be harnessed by the cell through the citric acid cycle. First,
pyruvate is decarboxylated to acetyl-CoA, an activated form of acetate ($CH_3COO^-$)

\[
    C_3H_3O_3^- + H{\mhyphen}SCoA + NAD^+ \rightarrow
    C_2H_3O{\mhyphen}SCoA + CO_2 + NADH
\]

This reaction is catalyzed by Pyruvate Dehydrogenase, a very complex enzyme with several
cofactors: lipoamide, FAD, coenzyme A. Thioester bond ($S{\mhyphen}C=O$) hydrolysis is very exergonic,
and therfore its formation demands energy. That energy comes from pyruvate decarboxylation
(pyruvate contains three carbon atoms, and the acetyl portion of acetyl-CoA only contains
two; the carboxylate group left as $CO_2$). Energy from decarboxylations is often used by the
cell to push an equilibrium towards product formation, as can be seen in several reactions
in the citric acid cycle and gluconeogenesis.

\subsection{The Krebs cycle}
The overall result of the Krebs cycle is to split Acetyl-CoA into its constituents, i.e.\
two molecules of $CO_2$ and the $CoA$ complex. In the process it consumes water $H_2O$ and
releases Hydrogen (protons and electrons). Balancing the reaction we get the following
simplified reaction:
\[
    C_2H_3O{\mhyphen}SCoA + 3H_2O \rightarrow 2CO_2 + HSCoA + 8H
\]
$HS{\mhyphen}$ is a thiol group. It is similar to an alcohol group, because $S$ is similar to
oxygen.

In practice, the hydrogen atoms are split intro protons and electrons, by the following
reaction
\[
    NAD^+ + 2H \rightarrow NADH + H^+
\]
$NAD^+$ is oxidized, $NADH$ is reduced.

In the following I'll describe each step in the Krebs cycle, and finally give a summary of
all the parts in the reaction.

\subsubsection{Citrate Synthase}
Acetyl-CoA reacts with Oxaloacetate to generate Citrate in an aldol addition.
Thioester hydrolysis helps to displace equilibrium towards product formation:
\[
    C_2H_3O{\mhyphen}SCoA + C_4H_2O_5^{2-} + H_2O \rightarrow
    C_6H_6O_7^{2-} + H{\mhyphen}SCoA
\]

\subsubsection{Aconitase}
Water is removed from Citrate to generate Aconitate.
\[
    C_6H_6O_7^{2-} \rightarrow C_6H_4O_6^{2-} + H_2O
\]

Then water is re-added to generate Isocitrate.
\[
    C_6H_4O_6^{2-} + H_2O \rightarrow C_6H_6O_7^{2-}
\]

\subsubsection{Isocitrate Dehydrogenase}
This converts Isocitrate into alpha-ketoglutarate.
\[
    C_6H_6O_7^{2-} + NAD^+ \rightarrow C_5H_4O_5^{2-} + CO_2 + NADH + H^+
\]

\subsubsection{Alpha-ketoglutarate Dehydrogenase}
alpha-ketoglutarate is an alpha-ketoacid, i.e., it contains a carbonyl group adjacent to a
carboxylic acid. We can predict that it will react like pyruvate, i.e., that its
decarboxylation may yield enough energy to enable the formation of a thioester bond with
coenzyme A. And this indeed occurs \ldots The enzyme involved (alpha-ketoglutarate dehydrogenase),
is quite similar to pyruvate dehydrogenase in composition, cofactors and mechanism.

\[
    C_5H_4O_5^{2-} + H{\mhyphen}SCoA + NAD^+ \rightarrow
    C_4H_4O_3^{-}{\mhyphen}SCoA + CO_2 + NADH
\]
This converts alpha-ketoglutarate into Succinyl-CoA.

\subsubsection{Succinyl-CoA Synthetase}
Like every thioester bond, the one present in succinyl-CoA is quite energetic. Its
hydrolysis will be the only step in the citric acid cycle where direct production of ATP
(or equivalent) occurs.
\[
    C_4H_4O_3^{-}{\mhyphen}SCoA + GDP^{3-} + HPO_4^{2-} \leftrightarrow
    C_4H_4O_4^{2-} + H{\mhyphen}SCoA + GTP^{4-}
\]
This converts Succinyl-CoA into Succinate.

The remaining steps are to convert the four-carbon molecule Succinate $C_4H_4O_4^{2-}$
into the four-carbon molecule Oxaloacetate $C_4H_2O_5^{2-}$. These remaining steps
essentially add a water molecule $H_2O$ to bring in the missing oxygen, and then remove
the excess hydrogen using $FAD$ and $NAD$.

Like Oxaloacetate, Succinate is a four-carbon product. The last reactions of the citric
acid cycle will regenerate Oxaloacetate from Succinate. Succinate is first oxidized to
Fumarate, by the Succinate Dehydrogenase Complex (also known as Complex II), which is
present in the matrix side of the inner mitochondrial membrane. The redox potential of the
oxidation of a C-C single bond to a C=C double bond (alkanes to alkenes) is too high to
enable the involved electrons to be accepted by NAD+ ($E_0$=-320 mV). The cell will therefore
use FAD ($E_0$= 0 mV) as electron acceptor. Fumarate hydration yields Malate, which can be
oxidized to Oxaloacetate, thus closing the cycle.

\subsubsection{Succinate Dehydrogenase}
\[
    C_4H_4O_4^{2-} + FAD \leftrightarrow C_4H_2O_4^{2-} + FADH_2
\]
This converts Succinate into Fumarate.

\subsubsection{Fumarase}
\[
    C_4H_2O_4^{2-} + H_2O \leftrightarrow C_4H_4O_5^{2-}
\]
This converts Fumarate into Malate.

\subsubsection{Malate Dehydrogenase}\label{sec_malate_dehydrogenase}
\[
    C_4H_4O_5^{2-} + NAD^+ \leftrightarrow C_4H_2O_5^{2-} + NADH + H^+
\]
This converts Malate into Oxaloacetate.

We're finally back where we started, replenishing the Oxaloacetate we used initially.
Except now the Acetyl-CoA has been broken down into $CO_2$ and Hydrogen.

\subsubsection{Total}
The total reaction is as follows:
\[
    \begin{split}
    C_2H_3O{\mhyphen}SCoA + GDP^{3-}+ HPO_4^{2-} + FAD + 3NAD^+ + 2H_2O \rightarrow \\
    2CO_2 + HSCoA + GTP^{4-} + FADH_2 + 3NADH + 2H^+
    \end{split}
\]
Note that the $GTP$ contains an extra $PO_3$ group compared to $GDP$.

Note also that the two specific carbon atoms in the generated $CO_2$ molecule do not come
from the Acetyl-CoA. So there is a re-shuffling of carbon atoms occurring in the Krebs
cycle.

\subsection{Oxidative Phosphorylation}
This is also know as the electron transport chain.

The flow of electrons through the electron transport chain is an exergonic process. The
energy from the redox reactions creates an electrochemical proton gradient that drives the
synthesis of adenosine triphosphate (ATP). In aerobic respiration, the flow of electrons
terminates with molecular oxygen as the final electron acceptor. In anaerobic respiration,
other electron acceptors are used, such as sulfate.

In an electron transport chain, the redox reactions are driven by the difference in the
Gibbs free energy of reactants and products. The free energy released when a higher-energy
electron donor and acceptor convert to lower-energy products, while electrons are
transferred from a lower to a higher redox potential, is used by the complexes in the
electron transport chain to create an electrochemical gradient of ions. It is this
electrochemical gradient that drives the synthesis of ATP via coupling with oxidative
phosphorylation with ATP synthase.

The energy released from one turn of the Krebs cycle is mainly stored in the reduced
molecules $FADH_2$ and $NADH$, as well as $GTP$. So, later in the metabolic pathway, when
these products are converted back into $FAD$, $NAD^+$, and $GDP$, this is where most of
the energy is released.

The Krebs cycle in itself does not require oxygen. However, replenishing $FAD$ and $NAD^+$
from the reduced $FADH_2$ and $NADH$ is an oxidation process that requires electron
acceptors such as oxygen.

$NAD^+$ is oxidized $NAD$, it can receive two electrons.
Whereas $NADH$ is reduced, it can donate electrons.

A {\em reduced\/} substrate has all its electrons intact, and can act as electron donor.
A {\em oxidized\/} substrate has lost some of its electrons, and can act as electron
acceptor.

\[
    {\rm reduced substrate} + NAD^+ \rightarrow {\rm oxidized substrate} + NADH + H^+
\]
The substrate in this reaction will be oxidized, i.e.\ it loses two electrons and two
protons, i.e.\ two hydrogen atoms. Similarly, the $NAD^+$ will be reduced, i.e.\ it
accepts two electrons and one proton (aka a hydride anion).

See section~(\ref{sec_malate_dehydrogenase}), where Malate is oxidized to become
Oxaloacetate.

Because $NADH$ is reduced it contains a lot of stored energy and it can act as electron
donor, and it therefore appears at the start of the Electron Transport Chain.

\subsubsection{NADH Dehydrogenase}
This is also called ``Complex I'' or ``NADH/uniquinone oxidoreductase''.

The reduced NADH will be oxidixed (thus losing its two extra electrons) and in turn
the ubiqionone molecule will be reduced (and gaining the two electrons).
In effect, two carbonyl groups ($=O$) are replaced by corresponding alcohol groups
(${\mhyphen}OH$).

\[
    Q + NADH + H^+ \rightarrow QH_2 + NAD^+
\]
where $Q$ is Ubiquinone and $QH_2$ is Ubiquinol.

In other words, Complex 1 reduces Ubiquinone and oxidizes NADH\@.

The energy released in this process is used to ``pump'' four $H^+$ protons across the
inner mitoondrial membrane, i.e.\ against the proton gradient.

\subsubsection{Succinate Dehydroginase}
This is also called ``Complex II''. This is part of the Krebs cycle, where
Succinate is converted in Fumarate via this reaction:
\[
    C_4H_4O_4^{2-} + FAD \leftrightarrow C_4H_2O_4^{2-} + FADH_2
\]
In other words, Succinate is oxidized to Fumarate, and $FAD$ is reduced to $FADH_2$.

Then $FADH_2$ is re-oxidized and returns the $H_2$ via this reaction:
\[
    Q + FADH_2 \rightarrow QH_2 + FAD
\]

\subsubsection{Q-cycle}
This is also called ``Complex III''.

Ubiquinol ($QH_2$) has two electrons to give.
\[
    QH_2 \rightarrow QH_2^{2+} + 2e^-
\]

First, one of these electrons is donated to another molecule of Uniquinone ($Q$), causing the
formation of a benzene ring.  One oxygen becomes negatively charged, while the other
becomes a free radical, leading to a Semiquinone.
\[
    Q + e^- \rightarrow {}^{\bullet}Q^{-}
\]
The other electron is donated to Cytochrome C in the inner membrane space, which will later
pass it on to Complex IV, see later. Finally, the two protons will go into the inner
membrane space (again against the proton gradient). So the total reaction so far for
Ubiquinol is:
\[
    QH_2 + Q + C \rightarrow Q + 2H^+ + {}^{\bullet}Q^{-} + C^-
\]

Now, a second molecule of Ubiquinol ($QH_2$) enters and reacts with the Semiquinone.
It too will give one electron to Semiquinone, one to Cytochrome C, and two protons
are pumped into the inner membran space (again against the proton gradient).
\[
    QH_2 + {}^{\bullet}Q^{-} + C \rightarrow Q + 2H^+ + C^- + Q^{2-}
\]
Finally, two protons from the inner matrix react with the reduced Ubiquinone to give back
Ubiquinol:
\[
    Q^{2-} + 2H^+ \rightarrow QH_2
\]

Note that $C^-$ only carries an electron, not an additional proton.

The overall net result of this Q-cycle is the following:
\[
    QH_2 + 2H^+ + 2C \rightarrow Q + 4H^+ + 2C^-
\]

\subsubsection{Complex IV}
This is also called ``Complex IV''.
This reacts one molecule of oxygen ($O_2$) with four protons (from the matrix) and four
electrons (from Cytochrome C) to give two molecules of water. The energy released
from this reaction is used to pump four protons into the inner membrane space.
Overall, this can be written as:
\[
    8H^+ + O_2 + 4C^- \rightarrow 4C + 2H_2O + 4H^+
\]
This is the only step that is dependent on oxygen. However, the whole electron transport
chain is dependent on this step functioning.

\subsubsection{Summary}
The overall net result of the electron transport chain is to take the electrons from the
reduced $NADH$ and $FADH_2$ and donate them to oxygen, thereby releasing a large amount of
energy. The reaction is:
\[
    FADH_2 + NADH + H^+ + O_2 \rightarrow FAD + NAD^+ + 2H_2O
\]
The energy relased (from reducing oxygen) is used to pump up to twelve protons across the
membrane (against the proton gradient).


\subsection{Pyruvate Carboxylase}
The following reaction converts pyruvate into oxaloacetate, and is catalyzed by Pyruvate
Carboxylase.
\[
    {ATP}^{4-} + {C_3H_3O_3}^- + {HCO_3}^- \rightarrow
    {ADP}^{3-} + {HPO_4}^{2-} + {C_4H_3O_5}^-
\]

\subsection{Glycolysis}
\[
    \begin{split}
    C_6H_{12}O_6 + 2{NAD}^+ + 2{ADP}^{3-} + 2{HPO_4}^{2-} \rightarrow \\
    2{C_3H_3O_3}^- + 2NADH + 2H^+ + 2{ATP}^{4-} + 2H_2O
    \end{split}
\]

\subsection{Reverse Krebs cycle}
\href{https://en.wikipedia.org/wiki/Biological_carbon_fixation}{The reverse Krebs cycle},
also known as reverse TCA cycle (rTCA) or reductive citric acid
cycle, is an alternative to the standard Calvin-Benson cycle for carbon fixation. It has
been found in strict anaerobic or microaerobic bacteria (as Aquificales) and anaerobic
archea. It was discovered by Evans, Buchanan and Arnon in 1966 working with the
photosynthetic green sulfur bacterium Chlorobium limicola. In particular, it is one of
the most used pathways in hydrothermal vents by the Campylobacterota. This feature is
very important in oceans. Without it, there would be no primary production in aphotic
environments, which would lead to habitats without life. So this kind of primary
production is called ``dark primary production''.

The cycle involves the biosynthesis of acetyl-CoA from two molecules of $CO_2$. The key
steps of the reverse Krebs cycle are:

\begin{itemize}

    \item Oxaloacetate to malate, using NADH + H+\\
        $Oxaloacetate + NADH / H^+ \rightarrow Malate + NAD^+$
    \item Fumarate to succinate, catalyzed by an oxidoreductase, Fumarate
        reductase\\
        $Fumarate + FADH_2 \leftrightarrow Succinate + FAD$
    \item Succinate to succinyl-CoA, an ATP dependent step\\
        $Succinate + ATP + CoA \rightarrow Succinyl-CoA + ADP + Pi$
    \item Succinyl-CoA to alpha-ketoglutarate, using one molecule of $CO_2$\\
        $Succinyl-CoA + CO2 + Fd{(red)} \rightarrow alpha-ketoglutarate + Fd{(ox)}$
    \item Alpha-ketoglutarate to isocitrate, using NADPH + H+ and another molecule of
        $CO_2$\\
        $Alpha-ketoglutarate + CO_2 + NAD(P)H/H^+ \rightarrow Isocitrate + NAD{(P)}^+$
    \item Citrate converted into oxaloacetate and acetyl-CoA, this is an ATP
        dependent step and the key enzyme is the ATP citrate lyase\\
        $Citrate + ATP + CoA \rightarrow Oxaloacetate + Acetyl-CoA + ADP + Pi$
\end{itemize}

This pathway is cyclic due to the regeneration of the oxaloacetate.

The bacteria Gammaproteobacteria and Riftia pachyptila switch from the Calvin-Benson cycle
to the rTCA cycle in response to concentrations of H2S.

\subsection{The Wood–Ljungdahl pathway}
\href{https://en.wikipedia.org/wiki/Wood-Ljungdahl_pathway}{The Wood-Ljungdahl pathway} is
a set of biochemical reactions used by some bacteria. It is
also known as the reductive acetyl-coenzyme A (Acetyl-CoA) pathway. This pathway
enables these organisms to use hydrogen as an electron donor, and carbon dioxide as an
electron acceptor and as a building block for biosynthesis.

\subsection{Anaerobic respiration}
\href{https://en.wikipedia.org/wiki/Anaerobic_respiration}{Anaerobic respiration} is
respiration using electron acceptors other than molecular oxygen ($O_2$). Although oxygen
is not the final electron acceptor, the process still uses a respiratory electron
transport chain.

In aerobic organisms undergoing respiration, electrons are shuttled to an electron
transport chain, and the final electron acceptor is oxygen. Molecular oxygen is an
excellent electron acceptor. Anaerobes instead use less-oxidizing substances such as
nitrate ($NO_3^-$), fumarate ($C_4H_2O_4^{2-}$), sulfate ($SO_4^{2-}$), or elemental
sulfur ($S$). These terminal electron acceptors have smaller reduction potentials than
$O_2$.  Less energy per oxidized molecule is released. Therefore, anaerobic respiration is
less efficient than aerobic.

\subsection{Methanogenesis}
\href{https://en.wikipedia.org/wiki/Methanogenesis}{Methanogenesis}
is the formation of methane coupled to energy conservation by microbes known as
methanogens. Organisms capable of producing methane for energy conservation have been
identified only from the domain Archaea, a group phylogenetically distinct from both
eukaryotes and bacteria, although many live in close association with anaerobic bacteria.
The production of methane is an important and widespread form of microbial metabolism. In
anoxic environments, it is the final step in the decomposition of biomass. Methanogenesis
is responsible for significant amounts of natural gas accumulations, the remainder being
thermogenic.

Methanogenesis in microbes is a form of anaerobic respiration. Methanogens do not use
oxygen to respire; in fact, oxygen inhibits the growth of methanogens. The terminal
electron acceptor in methanogenesis is not oxygen, but carbon. The two best described
pathways involve the use of acetic acid or inorganic carbon dioxide as terminal electron
acceptors:
\[
    CO_2 + 4 H_2 \rightarrow CH_4 + 2 H_2O
\]
\[
    CH_3COOH \rightarrow CH_4 + CO_2
\]

Methanogenesis is the final step in the decay of organic matter. During the decay process,
electron acceptors (such as oxygen, ferric iron, sulfate, and nitrate) become depleted,
while hydrogen ($H_2$) and carbon dioxide accumulate. Light organics produced by fermentation
also accumulate. During advanced stages of organic decay, all electron acceptors become
depleted except carbon dioxide. Carbon dioxide is a product of most catabolic processes,
so it is not depleted like other potential electron acceptors.

Only methanogenesis and fermentation can occur in the absence of electron acceptors other
than carbon. Fermentation only allows the breakdown of larger organic compounds, and
produces small organic compounds. Methanogenesis effectively removes the semi-final
products of decay: hydrogen, small organics, and carbon dioxide. Without methanogenesis, a
great deal of carbon (in the form of fermentation products) would accumulate in anaerobic
environments.

\subsection{Ferredoxin}
\href{https://en.wikipedia.org/wiki/Ferredoxin}{Ferredoxin}
are iron–sulfur proteins that mediate electron transfer in a range of metabolic reactions.
The term ``ferredoxin'' was coined by D.C. Wharton of the DuPont Co.\ and applied to the
``iron protein'' first purified in 1962 by Mortenson, Valentine, and Carnahan from the
anaerobic bacterium Clostridium pasteurianum.

Another redox protein, isolated from spinach chloroplasts, was termed ``chloroplast
ferredoxin''. The chloroplast ferredoxin is involved in both cyclic and non-cyclic
photophosphorylation reactions of photosynthesis. In non-cyclic photophosphorylation,
ferredoxin is the last electron acceptor thus reducing the enzyme $NADP^+$ reductase. It
accepts electrons produced from sunlight-excited chlorophyll and transfers them to the
enzyme ferredoxin: $NADP^+$ oxidoreductase.

Ferredoxins are small proteins containing iron and sulfur atoms organized as iron–sulfur
clusters. These biological ``capacitors'' can accept or discharge electrons, with the
effect of a change in the oxidation state of the iron atoms between +2 and +3. In this
way, ferredoxin acts as an electron transfer agent in biological redox reactions.

Other bioinorganic electron transport systems include rubredoxins, cytochromes, blue
copper proteins, and the structurally related Rieske proteins.

Ferredoxins can be classified according to the nature of their iron–sulfur clusters and by
sequence similarity.

Ferredoxins are one of the most reducing biological electron carriers. They typically have
a mid point potential of -420 mV. The reduction potential of a substance in the cell
will differ from its midpoint potential depending on the concentrations of its reduced and
oxidized forms. For a one electron reaction, the potential changes by around 60 mV for
each power of ten change in the ratio of the concentration. For example, if the ferredoxin
pool is around 95\% reduced, the reduction potential will be around -500 mV. In
comparison, other biological reactions mostly have less reducing potentials: for example
the primary biosynthetic reductant of the cell, NADPH has a cellular redox potential of
-370 mV ($E_0$ = -320 mV).

The highly reducing ferredoxins are reduced either by using another strong reducing agent,
or by using some source of energy to ``boost'' electrons from less reducing sources to the
ferredoxin.

\subsubsection{Direct reduction}
Reactions that reduce Fd include the oxidation of aldehydes to acids like the
glyceraldehyde to glycerate reaction (-580 mV), the carbon monoxide dehydrogenase reaction
(-520 mV), and the 2-oxoacid:Fd Oxidoreductase reactions (-500 mV) like the
reaction carried out by pyruvate synthase.

\subsubsection{Membrane potential coupled reduction}
Ferredoxin can also be reduced by using NADH (-320
mV) or H 2 (-414 mV), but these processes are coupled to the consumption of the membrane
potential to power the ``boosting'' of electrons to the higher energy state. The Rnf
complex is a widespread membrane protein in bacteria that reversibly transfers electrons
between NADH and ferredoxin while pumping Na+ or H+ ions across the membrane. The
chemiosmotic potential of the membrane is consumed to power the unfavorable reduction of
$Fd ox$ by NADH\@. This reaction is an essential source of $Fd_{red}^-$ in many autotrophic
organisms. If the cell is growing on substrates that provide excess $Fd_{red}^-$, the Rnf
complex can transfer these electrons to NAD+ and store the resultant energy in the
membrane potential. The energy converting hydrogenases (Ech) are a family of enzymes
that reversibly couple the transfer of electrons between $Fd$ and $H_2$ while pumping $H^+$ ions
across the membrane to balance the energy difference.

\[
    Fd_{ox}^0 + NADH + Na_{outside}^+ \leftrightarrow Fd_{red}^{2-} + NAD^+ + Na_{inside}^+
\]
\[
    Fd_{ox}^0 + H_2 + H_{outside}^+ \leftrightarrow Fd_{red}^{2-} + H^+ + H_{inside}^+
\]

\subsubsection{Electron bifurcation}
The unfavourable reduction of Fd from a less reducing electron donor can be coupled
simultaneously with the favourable reduction of an oxidising agent through an electron
bifurcation reaction.[6] An example of the electron bifurcation reaction is the generation
of Fd− red for nitrogen fixation in certain aerobic diazotrophs. Typically in oxidative
phosphorylation the transfer of electrons from NADH to Ubiquinone (Q) is coupled to
charging the proton motive force. In Azotobacter the energy released by transferring one
electron from NADH to Q is used to simultaneously boost the transfer of one electron from
NADH to Fd.

\subsubsection{Direct reduction of high potential ferredoxins}
Some ferredoxins have a sufficiently high redox potential that they can be directly
reduced by NADPH\@. One such ferredoxin is adrenoxin (-274mV) which takes part in the
biosynthesis of many mammalian steroids.  The ferredoxin Fd3 in the roots of plants that
reduces nitrate and sulfite has a midpoint potential of -337mV and is also reduced by
NADPH\@.

%%%%%%%%%%%%%%%%%%%%%%%%%%%%%%%%%%%%%%%%%%%%%%%%%%%%%%%%%%%%

\pagebreak
\section{Notes}\label{sec_notes}

We need something about ``Standard enthalpy of formation''.

Writing chemical formulae with chemfig is straightforward.

This is based on \href{https://www.overleaf.com/learn/latex/Chemistry_formulae}{chemfig}.

Links:
\begin{itemize}
    \item\href{https://en.wikipedia.org/wiki/Biological_carbon_fixation}{Biological carbon fixation}.
    \item\href{https://courses.lumenlearning.com/wm-nmbiology1/chapter/citric-acid-cycle-and-oxidative-phosphorylation/}{citric
        acid cycle and oxidative phosphorylation}.
    \item\href{http://homepage.ufp.pt/pedros/bq/tca.htm}{The chemical logic behind \ldots}
\end{itemize}
\end{document}

