\documentclass{article}
\usepackage{chemfig}
\usepackage{hyperref}
\begin{document}

%%%%%%%%%%%%%%%%%%%%%%%%%%%%%%%%%%%%%%%%%%%%%%%%%%%%%%%%%%%%

\section{Introduction}

This document is about the bio-chemical processes happening in cells during metabolism and
respiration. The main entity in this process is the Krebs cycle.

The organisation is as follows: In section~(\ref{sec_prerequisites}) we begin by
establishing some facts from organic chemistry. Then in section~(\ref{sec_players})
we list the main characters taking place in this chemical dance.

%%%%%%%%%%%%%%%%%%%%%%%%%%%%%%%%%%%%%%%%%%%%%%%%%%%%%%%%%%%%

\section{Prerequisites}\label{sec_prerequisites}

Organic compounds are named after certain ``functional groups'' in the atomic structure.

\subsection{Alkane}

An alkane consists of carbon and hydrogen only, and with only single carbon-carbon bond.
They have the general chemical formula $C_{n}H_{2n+2}$. They are {\em saturated\/}
due to the lack of double bonds (and hence less reactive?).

Examples of alkanes are methane ($CH_4$), ethane ($C_2H_6$), and octane ($C_{8}H_{18}$).
Methane is produced by methanogenic bacteria.

\subsection{Alkene}

An alkene consists of carbon and hydrogen only, and with one carbon-carbon double bond.
They have the general chemical formula $C_{n}H_{2n}$. The double bond is stronger than
a single bond, but not twice as strong.

Alkenes react in many addition reactions, which occur by opening up the double-bond.

Examples of alkanes are ethylene ($C_2H_4$) and propylene ($C_3H_6$).

\subsection{Aldehyde}

\chemfig{R-C (=[1]O) (-[7]H)}

The functional group (O=C---H) is called {\em aldehyde group\/} but also sometimes {\em
formyl group}.

Examples of aldehydes are: formaldehyde, acetaldehyde, glucose. However, in aqueous
solution only a tiny fraction of glucose exists as the aldehyde.

\subsection{carboxylic acid}
\chemfig{R-C (=[1]O) (-[7]OH)}

Carboxylic acids are polar. Because they are both hydrogen-bond acceptors (the {\em
carbonyl\/} $–C=O$) and hydrogen-bond donors (the {\em hydroxyl\/} $–OH$), they also
participate in hydrogen bonding. Together, the hydroxyl and carbonyl group form the
functional group {\em carboxyl}.

Most carboxylic acids can be reduced to alcohols by hydrogenation.

Examples are formic acid (formate) and acetic acid (acetate).

\subsection{ester}
\chemfig{R-C (=[1]O) (-[7]OR')}

\subsection{ketone}
\chemfig{R-C (=[1]O) (-[7]R')}

\subsection{carbohydrate}
A carbohydrate is a biomolecule consisting of carbon (C), hydrogen (H) and oxygen (O)
atoms, usually with a hydrogen–oxygen atom ratio of 2:1 (as in water) and thus with the
empirical formula $C_m{(H_2O)}_n$ (where m may or may not be different from n), which does
not mean the H has covalent bonds with O (for example with $CH_2O$, H has a covalent bond
with C but not with O). However, not all carbohydrates conform to this precise
stoichiometric definition (e.g., uronic acids, deoxy-sugars such as fucose), nor are all
chemicals that do conform to this definition automatically classified as carbohydrates
(e.g.\ formaldehyde and acetic acid).

%%%%%%%%%%%%%%%%%%%%%%%%%%%%%%%%%%%%%%%%%%%%%%%%%%%%%%%%%%%%

\section{The players}\label{sec_players}

\subsection{Aldehydes}

\subsubsection{Formaldehyde}
\chemfig{H-C (=[1]O) (-[7]H)}

\begin{itemize}
\item Chemical formula $CH_2O$.
\item Standard enthalpy of formation: -108.7 kJ/mol.
\end{itemize}

\subsubsection{Acetaldehyde}
\chemfig{H-C (-[2]H) (-[6]H) -C (=[1]O) (-[7]H)}

\begin{itemize}
\item Chemical formula $C_2H_4O$.
\item Standard enthalpy of formation: -192.2 kJ/mol.
\end{itemize}

\subsection{Carboxylic acids}

In the following, the carboxylate anions are given, a result of deprotonation.

\subsubsection{Formate}
\chemfig{H-C (=[1]O)-[7]{O^-}}
\begin{itemize}
\item Chemical formula $CHO_2^-$.
\item Standard enthalpy of formation (of formic acid): -425.0 kJ/mol.
\end{itemize}
The two oxygen atoms are equivalent and bear a partial negative charge. The remaining C-H
bond is not acidic.

\subsubsection{Formate}
\[
C_2H_4O_2
\]
\chemfig{H-C (-[2]H) (-[6]H)-C (=[1]O)-[7]{O^-}}

%%%%%%%%%%%%%%%%%%%%%%%%%%%%%%%%%%%%%%%%%%%%%%%%%%%%%%%%%%%%

\section{Reactions}\label{sec_reactions}

bla bla

%%%%%%%%%%%%%%%%%%%%%%%%%%%%%%%%%%%%%%%%%%%%%%%%%%%%%%%%%%%%

\section{Notes}\label{sec_notes}

We need something about ``Standard enthalpy of formation''.

Writing chemical formulae with chemfig is straightforward.

This is based on \href{https://www.overleaf.com/learn/latex/Chemistry_formulae}{chemfig}.

Wikipedia links:
\href{https://en.wikipedia.org/wiki/Biological_carbon_fixation}{Biological carbon fixation}.
\end{document}

