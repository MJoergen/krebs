\documentclass[a4paper,14pt]{extarticle}
\usepackage{chemfig}
\usepackage{hyperref}
\usepackage{amsmath}
\usepackage{booktabs}
\usepackage[show]{ed}
\def\mhyphen{{\hbox{-}}}
\def\Al{{\rm Al}}
\def\H{{\rm H}}
\def\O{{\rm O}}
\def\M{{\rm M}}
\def\C{{\rm C}}
\def\Ca{{\rm Ca}}
\def\Fe{{\rm Fe}}
\def\Mg{{\rm Mg}}
\def\Mn{{\rm Mn}}
\def\Ni{{\rm Ni}}
\def\Si{{\rm Si}}
\def\II{{\rm II}}
\def\III{{\rm III}}
\begin{document}

%%%%%%%%%%%%%%%%%%%%%%%%%%%%%%%%%%%%%%%%%%%%%%%%%%%%%%%%%%%%

\section{Introduction}

This is about energy calculations in Redox reactions.
The following are the ``Standard Enthalpy of Formation'' and ``Standard Entropy'' of
various compounds. Units are SI, i.e. $\text{kJ}/\text{mol}$ and $\text{J}/(\text{mol}\cdot \text{K})$.
Source of data is~\cite{nist}.

\[
\begin{array}{rrrr}
    \text{Name}      & \text{Enthalpy} & \text{Entropy} & \text{Gibbs} \\
   \midrule
    \C_2\H_4\O_2 \,(l)  &  -483.52 & 158.00  &  -530.9 \\
    \C\H_2\O_2 \,(l)    &  -425.09 & 131.84  &  -464.6 \\
    \C\H_4 \,(g)        &   -74.60 & 186.25  &  -130.5 \\
    \C\O_2 \,(g)        &  -393.52 & 213.79  &  -457.7 \\
    \C\O \,(g)          &  -110.53 & 197.66  &  -169.8 \\
    \Fe_2\O_3 \,(s)     &  -825.50 &  87.42  &  -851.7 \\
    \Fe_3\O_4 \,(s)     & -1120.89 & 145.30  & -1164.5 \\
    \Fe{(\O\H)}_2 \,(s) &  -574.04 &  87.93  &  -600.4 \\
    \Fe{(\O\H)}_3 \,(s) &  -832.62 & 104.56  &  -864.0 \\
    \Fe\O \,(s)         &  -272.04 &  60.75  &  -290.3 \\
    \H_2\O \,(l)        &  -285.83 &  69.95  &  -306.8 \\
    \H_2\O \,(g)        &  -241.83 & 188.84  &  -298.5 \\
    \Mg_2\Si\O_4 \,(s)  & -2176.94 &  94.96  & -2205.4 \\
    \Mg\C\O_3 \,(s)     & -1111.69 &  65.84  & -1131.4 \\
    \Mg{(\O\H)}_2 \,(s) &  -924.66 &  63.18  &  -943.6 \\
    \Si\O_2 \,(s)       &  -910.86 &  41.47  &  -923.3 \\
    \H_2 \,(g)          &          & 130.68  &   -39.2 \\
    \O_2 \,(g)          &          & 205.15  &   -61.5
\end{array}
\]

%%%%%%%%%%%%%%%%%%%%%%%%%%%%%%%%%%%%%%%%%%%%%%%%%%%%%%%%%%%%

\section{Example}

Consider the reaction
\[
    2 \,\H_2 \,(g) + \O_2 \,(g) \rightarrow 2 \,\H_2\O \,(l).
\]
We find in the table that
\[
    \begin{array}{rcl}
        \Delta H &=& 2 \cdot -285.83 = -571.66 \,\text{kJ}/\text{mol} \\
        \Delta S &=& 2 \cdot 69.95 - 2 \cdot 130.68 - 205.15 = -326.61 \,\text{J}/(\text{mol}\cdot \text{K}).
    \end{array}
\]
From here we get (using $T = 300 \,\text{K} = 0.3 \,\text{kK}$)
\[
    \Delta G = -571.66 - 0.3 \cdot (-326.61) = -473.7 \,\text{kJ}/\text{mol}.
\]
So each mole of water generated releases half that amount, i.e. $237 \,\text{kJ}$ of energy.

As a second example consider the reaction
\[
    \C\H_4 \,(g)+ \H_2\O \,(g)\rightarrow \C\O \,(g)+ 3\,\H_2\,(g)
\]
Here Carbon is being oxidized (from $-4$ to $+2$) and Hydrogen is being reduced (from $+1$ to $0$).
From the table above we find
\[
    \Delta H = 74.69 + 241.83 - 110.53 - 3\cdot 0 = 206 \,\text{kJ}/\text{mol},
\]
and
\[
    \Delta G = 130.5 + 298.5 - 169.8 - 3\cdot 39.2 = 142 \,\text{kJ}/\text{mol}.
\]
So this reaction is endothermic.

%%%%%%%%%%%%%%%%%%%%%%%%%%%%%%%%%%%%%%%%%%%%%%%%%%%%%%%%%%%%

\section{Iron}
Iron, like Sulfur and Manganese, occurs in more than one oxidation state and therefore forms a variety of different
oxides.  Iron thus exists as native Iron, ferrous Iron ($\Fe^{\II}$) and ferric Iron ($\Fe^{\III}$).

The oxides of Iron are:

\begin{itemize}
    \item Iron (III) oxide: $\Fe_2\O_3$, aka Hematite.
    \item Iron (II,III) oxide: $\Fe_3\O_4$, aka Magnetite.
    \item Iron (II) oxide: $\Fe\O$, aka Wüstite.
    \item Iron (II) hydroxide: $\Fe{(\O\H)}_2$.
    \item Iron (III) hydroxide: $\Fe{(\O\H)}_3$, aka Limonite or Goethite.
\end{itemize}

\subsection{Iron (III) oxide}
This is also known as ``ferric oxide'' and has the stoichiometric formula $\Fe^{\III}_2\O_3$.
The crystalline structure $\alpha\mhyphen\Fe_2\O_3 $ occurs naturally as the mineral Hematite.
Its hydrated form, i.e.\ hydrous ferric oxide is often called rust.

\[
    \Fe^\III_2\O_3 \cdot \H_2\O
\]

\subsubsection{Thermite}
\[
    2\,\Al + \Fe_2\O_3 \rightarrow 2 \, \Fe + \Al_2\O_3
\]

\subsection{Iron (II,III) oxide}
This is magnetite and has the chemical composition
$\Fe^{\II}\O\cdot\Fe^{\III}_2\O_3$.

\subsection{Iron (II) oxide}
Iron (II) oxide, aka ferrous oxide, has the chemical composition $\Fe^{\II}\O$ also known as wüstite.

It may be generated from magnetite in the following reaction:
\[
    \Fe^\II\O\cdot\Fe^\III_2\O_3 + \C \rightarrow 3\,\Fe^\II\O + \C\O
\]
where Iron is reduced (from +3 to +2) and Carbon is oxidized (from 0 to +2).

The following decomposition reaction is exothermic.
\[
    4\,\Fe^\II\O \rightarrow \Fe + \Fe_3\O_4
\]
In this reaction, one iron atom is reduced twice (from +2 to 0), while two other iron atoms are oxidixed (from +2 to
+3).

\subsection{Iron (II) hydroxide}
Iron (II) hydroxide, aka ferrous hydroxide, is a crystal with the chemical formula $\Fe{(\O\H)}_2$.

The Schikorr reaction is
\[
    3 \,\Fe {(\O\H)}_2 \rightarrow \Fe_3\O_4 + \H_2 + 2\,\H_2\O
\]
where ferrous hydroxide (in anaerobic conditions) is oxidized by the protons of water to form magnetite (iron (II,III)
oxide).

\subsection{Iron (III) oxide-hydroxide}
Iron (III) oxide-hydroxide, aka ferric oxyhydroxide, has the formula $\Fe^\III\O(\O\H)$. The monohydrated form
$\Fe^\III\O(\O\H)\cdot \H_2\O$ is often abbreviated as $\Fe^\III{(\O\H)}_3$. This is one form of rust, the other being
Iron (III) oxide.

It is formed when Iron (II) hydroxide is exposed to air:
\[
    4 \,\Fe^\II{(\O\H)}_2 + \O_2 \rightarrow 4 \,\Fe^\III\O(\O\H) + 2 \,\H_2\O
\]
Here Iron is oxidized (from +2 to +3) and Oxygen is reduced (from 0 to -2).

\subsection{Redox buffer}
Magnetite and Hematite combine to form a redox buffer:
\[
    4 \,\Fe_3\O_4 + \O_2 \leftrightarrow 6\, \Fe_2\O_3
\]
where Iron is oxidized (from +2 to +3) and Oxygen is reduced (from 0 to -2).

Wüstite and Magnetite combine to form this redox buffer:
\[
    6 \,\Fe\O + \O_2 \leftrightarrow 2\,\Fe_3\O_4,
\]
where again Iron is oxidized (from +2 to +3) and Oxygen is reduced (from 0 to -2).

Fayalite and Magnetite combine to form this redox buffer:
\[
    3 \,\Fe_2\Si\O_4 + \O_2 \leftrightarrow 2 \,\Fe_3\O_4 + 3 \,\Si\O_2
\]
where yet again Iron is oxidized (from +2 to +3) and Oxygen is reduced (from 0 to -2).

\subsection{Iron furnace}
To extract iron, one starts with iron ore containing the naturally occurring mineral hematite ($\Fe_2\O_3$). The iron
atoms in the mineral must be reduced (from +3 to 0) and we therefore add Carbon as a reducing agent. However, pure Carbon is not
strong enough as a reducing agent, so we first convert it into Carbon Monoxide in a two-step process:
\[
    \C\,(s) + \O_2\,(g) \rightarrow \C\O_2\,(g)
\]
and
\[
    \C\O_2\,(g) + \C\,(s) \rightarrow 2\,\C\O\,(g)
\]
The Carbon Monoxide is a much strong reducing agent. Both of the above reactions are exothermic and supply much the heat
necessary for the next reaction.

The main redox reaction is
\[
    \Fe_2\O_3\,(s) + 3\,\C\O\,(g) \rightarrow 2\,\Fe\,(l) + 3\,\C\O_2\,(g)
\]
where Iron is reduced from +3 to 0 and Carbon is oxidized from +2 to +4.

\begin{thebibliography}{9}
    \bibitem{nist}              \href{https://webbook.nist.gov/}{NIST Chemistry WebBook}
    \bibitem{olivine}           \href{https://en.wikipedia.org/wiki/Olivine}{Olivine}
    \bibitem{orthosilicate}     \href{https://en.wikipedia.org/wiki/Orthosilicate}{Orthosilicate}
    \bibitem{mafic}             \href{https://en.wikipedia.org/wiki/Mafic}{Mafic}
    \bibitem{hydrated_silica}   \href{https://en.wikipedia.org/wiki/Hydrated_silica}{Hydrated Silica}
    \bibitem{quartz}            \href{https://en.wikipedia.org/wiki/Quartz}{Quartz}
    \bibitem{serpentinization}  \href{https://en.wikipedia.org/wiki/Serpentinization}{Serpentinization}
    \bibitem{schikorr_reaction} \href{https://en.wikipedia.org/wiki/Schikorr_reaction}{Schikorr reaction}
    \bibitem{iron_ii_iii_oxide} \href{https://en.wikipedia.org/wiki/Iron(II,III)_oxide}{Iron (II,III) oxide}
\end{thebibliography}

\end{document}

