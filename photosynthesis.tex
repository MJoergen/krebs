\documentclass[a4paper,14pt]{extarticle}
\usepackage{chemfig}
\usepackage{hyperref}
\usepackage{amsmath}
\usepackage{booktabs}
\usepackage[show]{ed}
\def\mhyphen{{\hbox{-}}}
\def\Al{{\rm Al}}
\def\H{{\rm H}}
\def\O{{\rm O}}
\def\M{{\rm M}}
\def\N{{\rm N}}
\def\C{{\rm C}}
\def\Ca{{\rm Ca}}
\def\Fe{{\rm Fe}}
\def\Mg{{\rm Mg}}
\def\Mn{{\rm Mn}}
\def\Ni{{\rm Ni}}
\def\S{{\rm S}}
\def\Si{{\rm Si}}
\def\II{{\rm II}}
\def\III{{\rm III}}
\begin{document}

%%%%%%%%%%%%%%%%%%%%%%%%%%%%%%%%%%%%%%%%%%%%%%%%%%%%%%%%%%%%

\section{Introduction}

Photosynthesis needs a source of Hydrogen gas.
Photosynthesis proceeds in two steps:
\begin{enumerate}
    \item Generate Hydrogen gas from some Hydrogen source, possibly using light as a helper.
    \item React Hydrogen gas with $\C\O_2$ to generate organic molecules.
\end{enumerate}
The first part requires two things:
\begin{itemize}
    \item A Hydrogen source, e.g. $\H_2\O$ or $\H_2\S$.
    \item An energy source.
    \item An electron source?
\end{itemize}
This is achieved with Chlorophyll.

The second part is achieved with Rubisco as a catalyst.

\section{Chlorophyll}

Chlorophyll acts like a catalyst; it absorbs photons at around 680 nm.

Some possible reactions
\[
    \H_2 + \C\O_2 + \text{energy} \rightarrow \C\H_2\O_2
\]
and
\[
    2\,\H_2\O + \C\O_2 + \text{energy} \rightarrow \H_2\O + \C\H_2\O + \O_2
\]
and
\[
    2\,\H_2\S + \C\O_2 + \text{energy} \rightarrow \H_2\O + \C\H_2\O + 2\,\S
\]
The last equation above likely came first in evolution. It requires less energy (about 2/3) to split $\H_2\S$
than to split $\H_2\O$.

So early evolution used Hydrogen Sulfide $\H_2\S$ as the source of Hydrogen, and only later did cells evolve the
capacity to split Water $\H_2\O$. The latter is abundant and therefore presented a huge advantage. Using Hydrogen
Sulfide is characteristic of Chemotrophs\ \cite{lithoautotroph}.

The equations above are examples of Oxygenic Photosynthesis. There is also Anoxygenic Photosynthesis. The latter
is much older\ \cite{anoxygenicphotosynthesis} and\ \cite{purplesulfurbacteria}.

Bacteriorhodopsin has a $\Fe$ or $\Mg$ core.

However, splitting of water (and in general any harvesting of photons) is very powerful machinery. The photons can
potentially split anything.

Ferri-cyanide $\Fe{(\C\N)}_6^{3-}$ can be used instead of $\C\O_2$.

\section{Electron Transport Chain}
\[
    \text{Chlorophyll} + \text{energy} \rightarrow \text{Chlorophyll}^+ + e^-
\]

Reaction Center Complex is used in the Electron Transport Chain.

The electron $e^-$ moves from compounds with low redox potential to high potential. During this process
the excited electron facilitates the generation of ADP.\@

The excited Chlorophyll reacts with (i.e.\ steals electrons from) water.
In other words, the water molecule $\H_2\O$ or rather the $\H_2$ acts as an electron donor.

This is known as the Z-scheme.

\section{Catalase}
An enzyme (containing either Iron or Manganese at its core) that breaks down Hydrogen Peroxide into Water and Oxygen.
\[
    2\,\H_2\O_2 \rightarrow 2\,\H_2\O + \O_2
\]

This uses Hydrogen Peroxide as an electron donor instead of water and Sulfur as an electron acceptor instead of
Oxygen\ \cite{earlyanaerobicmetabolisms}.

In the early Earth, the water was full of Iron and no Oxygen. There was some amount of Hydrogen Peroxide and that lead
to generation of radicals via the two Fenton reactions (where Iron ions act as a catalyst):
\[
    \Fe^{2+} + \H_2\O_2 \rightarrow \Fe^{3+} + \H\O^{*} + \O\H^-
\]
and
\[
    \Fe^{3+} + \H_2\O_2 \rightarrow \Fe^{2+} + \H\O\O^{*} + \H^+
\]
They convert Hydrogen Perodixe $\H_2\O_2$ into a hydroxyl radical $\H\O^*$ and a hydroperoxyl radical
$\H\O\O^*$.

Hydrogen Peroxide is generated in (among other places) the atmosphere when UV light reacts with water:
\[
    \H_2\O + \text{photon} \rightarrow \H\O^* + e^- + \H^+
\]
and
\[
    2\,\H\O^* \rightarrow \H_2\O_2
\]

Once Oxygen became abundant, the Iron got oxidized and precipitated out of the oceans, hence reducing the abundance of
free radicals.

Two Catalase molecules can combine into one single Oxygen-evolving complex\ \cite{oxygenevolvingcomplex}.


\section{Oxygenic photosynthesis}
This evolved due to the selective pressure from lack of Iron and Hydrogen Sulfides and other minerals,
as well as ultraviolet radiation causing production of Hydrogen Peroxide.

And it required already existing Catalase as a necessary precursor.

Chlorophyll:\ \cite{lightdependentreactions}
The overall reaction is as follows:
\[
    2\,\H_2\O + 8 \, \text{photons} \rightarrow 4\,e^- + 4\,\H^+ + \O_2,
\]
but it proceeds in a number of part reactions (one for each photon).
In other words, water is stripped of its electrons (four in total) one by one, in order to produce a single Oxygen
molecule.
It takes place in the Oxygen-evolving complex\ \cite{oxygenevolvingcomplex} and\ \cite{catalase}.

Oxygen (even in molecular form) are very reactive and tend to steal electrons to form Super Oxide Radicals.
\[
    2\,\H^+ + 2\,e^- + \C\O_2 \rightarrow \C\H_2\O_2.
\]

\begin{thebibliography}{9}
    \bibitem{lithoautotroph}            \href{https://en.wikipedia.org/wiki/Lithoautotroph}{Lithoautotroph}
    \bibitem{anoxygenicphotosynthesis}  \href{https://en.wikipedia.org/wiki/Anoxygenic_photosynthesis}{Anoxygenic photosynthesis}
    \bibitem{lightdependentreactions}   \href{https://en.wikipedia.org/wiki/Light-dependent_reactions}{Light-dependent reactions}
    \bibitem{purplesulfurbacteria}      \href{https://en.wikipedia.org/wiki/Purple_sulfur_bacteria}{Purple sulfur bacteria}
    \bibitem{oxygenevolvingcomplex}     \href{https://en.wikipedia.org/wiki/Oxygen-evolving_complex}{Oxygen-evolving complex}
    \bibitem{catalase}                  \href{https://en.wikipedia.org/wiki/Catalase}{Catalase}
    \bibitem{earlyanaerobicmetabolisms} \href{https://pmc.ncbi.nlm.nih.gov/articles/PMC1664682/}{Early anaerobic metabolisms}
\end{thebibliography}

\end{document}

