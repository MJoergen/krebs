\documentclass[a4paper,14pt]{extarticle}
\usepackage{chemfig}
\usepackage{hyperref}
\usepackage{amsmath}
\usepackage{booktabs}
\usepackage[show]{ed}
\def\mhyphen{{\hbox{-}}}
\def\Al{{\rm Al}}
\def\H{{\rm H}}
\def\O{{\rm O}}
\def\M{{\rm M}}
\def\N{{\rm N}}
\def\C{{\rm C}}
\def\Ca{{\rm Ca}}
\def\Fe{{\rm Fe}}
\def\Mg{{\rm Mg}}
\def\Mn{{\rm Mn}}
\def\Ni{{\rm Ni}}
\def\S{{\rm S}}
\def\Si{{\rm Si}}
\def\II{{\rm II}}
\def\III{{\rm III}}
\begin{document}

%%%%%%%%%%%%%%%%%%%%%%%%%%%%%%%%%%%%%%%%%%%%%%%%%%%%%%%%%%%%

\section{Introduction}

Photosynthesis needs a source of Hydrogen gas.
Photosynthesis proceeds in two steps:
\begin{enumerate}
    \item Generate Hydrogen gas from some Hydrogen source, possibly using light as a helper.
    \item React Hydrogen gas with $\C\O_2$ to generate organic molecules.
\end{enumerate}
The first part requires two things:
\begin{itemize}
    \item A Hydrogen source, e.g. $\H_2\O$ or $\H_2\S$.
    \item An energy source.
    \item An electron source?
\end{itemize}
This is achieved with Chlorophyll.

The second part is achieved with Rubisco as a catalyst.

\section{Chlorophyll}

Chlorophyll acts like a catalyst; it absorns photons at around 680 nm.

Some possible reactions
\[
    \H_2 + \C\O_2 + \text{energy} \rightarrow \C\H_2\O_2
\]
and
\[
    2\,\H_2\O + \C\O_2 + \text{energy} \rightarrow \H_2\O + \C\H_2\O + \O_2
\]
and
\[
    2\,\H_2\S + \C\O_2 + \text{energy} \rightarrow \H_2\O + \C\H_2\O + 2\,\S
\]
The last equation above likely came first in evolution. It requires less energy (about 2/3) to split $\H_2\S$
than to split $\H_2\O$.

So early evolution used Hydrogen Sulfide $\H_2\S$ as the source of Hydrogen, and only later did
cells evolve the capacity to split Water $\H_2\O$. The latter is abundant and therefore presented a huge advantage.

However, splitting of water (and in general any harvesting of photons) is very powerful machinery. The photons
can potentially split anything.

Ferri-cyanide $\Fe{(\C\N)}_6^{3-}$ can be used instead of $\C\O_2$.

\section{Electron Transport Chain}
\[
    \text{Chlorophyll} + \text{energy} \rightarrow \text{Chlorophyll}^+ + e^-
\]

The electron $e^-$ moves from compounds with low redox potential to high potential. During this process
the excited electron facilitates the generation of ADP.\@

The excited Chlorophyll reacts with (i.e.\ steals electrons from) water.
In other words, the water molecule $\H_2\O$ acts as an electron donor.

This is known as the Z-scheme.

\section{Rubisco}
\[
    \H^+ + e^- \rightarrow \H {}
\]

\begin{thebibliography}{9}
    \bibitem{nist}              \href{https://webbook.nist.gov/}{NIST Chemistry WebBook}
    \bibitem{olivine}           \href{https://en.wikipedia.org/wiki/Olivine}{Olivine}
    \bibitem{orthosilicate}     \href{https://en.wikipedia.org/wiki/Orthosilicate}{Orthosilicate}
    \bibitem{mafic}             \href{https://en.wikipedia.org/wiki/Mafic}{Mafic}
    \bibitem{hydrated_silica}   \href{https://en.wikipedia.org/wiki/Hydrated_silica}{Hydrated Silica}
    \bibitem{quartz}            \href{https://en.wikipedia.org/wiki/Quartz}{Quartz}
    \bibitem{serpentinization}  \href{https://en.wikipedia.org/wiki/Serpentinization}{Serpentinization}
    \bibitem{schikorr_reaction} \href{https://en.wikipedia.org/wiki/Schikorr_reaction}{Schikorr reaction}
    \bibitem{iron_ii_iii_oxide} \href{https://en.wikipedia.org/wiki/Iron(II,III)_oxide}{Iron (II,III) oxide}
\end{thebibliography}

\end{document}

